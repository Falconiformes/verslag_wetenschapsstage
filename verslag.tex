\documentclass[abstract=true]{scrartcl}
\usepackage{lmodern}
\usepackage{microtype}
\usepackage[dutch]{babel}
\usepackage[utf8]{inputenc}
\usepackage{csquotes}
%%% Font & taal
\usepackage{booktabs,array,multirow,tabularx} %% Tabellen libraries
\usepackage{siunitx}
\usepackage[version=4]{mhchem}
%%% Figuur programmeren
\usepackage{tikz}
\usetikzlibrary{positioning,calc,intersections,matrix}
\newcommand{\Receptor}[1]{%
\begin{scope}[rotate around={#1:(0,0)}]
    \draw[%
    ultra thick,
    smooth,
    line cap=round
    ] (-.50em,1em) -- ++(0,2.5em) 
    to[bend left] ++(-.5em,1.25em) to[bend left] ++ (-1.0em,1.25em);
%
    \draw[%
    xscale=-1,
    ultra thick,
    smooth,
    line cap=round
    ] (-.50em,1em) -- ++(0,2.5em) 
    to[bend left] ++(-.5em,1.25em) to[bend left] ++ (-1.0em,1.25em);
\end{scope}
}
\newcommand{\cellmembrane}[2]{%
        \node[fill=#1!45,
        minimum size=5em,
        inner sep=0,
        draw,
        circle,ultra thick
        ] (#2) at (0,0) {};
}
\newcommand{\nucleus}[2]{%
        \node[%
        shading=ball,
        ball color=#1!50,
        minimum size=2.5em,
        inner sep=0,
        draw,
        circle,
        ultra thick
        ] (#2) at (-0.20,0.20) {};
}
\newcommand{\flowblock}[4][]{%
    \node[%
        inner sep=5,
        minimum width=4cm,
        draw,
        fill=blue!25,
        align=center,
        rounded corners,
        #1
                     ]
                     (#2) {#3\\ ($n = #4$)};
}
\tikzset{>=latex}
\usepackage{graphicx}
%%% Marges bepaling
\usepackage{biblatex}
\addbibresource{Bronnen/bib.bib}
%%% Bibliografie
\usepackage[pdfusetitle]{hyperref}
\hypersetup{%
    colorlinks=true,
    citecolor=blue,
    pdfauthor={Edon Namani; Martijn Schuiling; Gijs Stuart; Casper Jansen; Martein Leen},
    pdftitle={Effect van borstvoeding op ontwikkelingskans van astma},
    pdfsubject={Immunologie; Allergie},
    pdfkeywords={Immunologie; Allergie; Borstvoeding; Astma; Odds ratio}
}
\usepackage[dutch]{cleveref}
%%% Hyperlinks & easy referencing
\usepackage[acronym,smallcaps]{glossaries}
\usepackage{scrlayer-scrpage}
\ihead{Borstvoeding \& astma}
\ifoot{\copyright{} 2019 \LaTeXe{} foundations}
\usepackage{threeparttable}
\addtokomafont{captionlabel}{\bfseries}
\KOMAoptions{DIV=calc}
%%%% acronyms
\newacronym{lovk}{lovk}{lange onverzadigde vetzuurketen}
\makeglossaries

\title{Effect van borstvoeding op ontwikkelingskans van astma}
\subtitle{Wetenschapsstage}
\author{Edon Namani\thanks{Praeses} \and Martijn Schuiling\thanks{Co\"ordinator} \and Gijs Stuart \and Casper Jansen \and Martein Leen}
\date{\today}
\subject{Immunologie \& allergie}
\begin{document}
\maketitle
    \begin{abstract}

    \end{abstract}

\section{Introductie}
In de afgelopen decennia is de incidentie van astma en andere type I hypersensitiviteit morbi ferm toegenomen \cite{Platts_Mills_2015}. Deze toegenomen incidentie veroorzaakt een verhoogde zorglast voor de maatschappij. Astma is een kinderlijke chronische respiratoire morbus. Meerdere factoren dragen bij aan de ontwikkeling van astma. De factoren zijn de genetische predispositie, de wijze van bevalling en het dieet \cite{abbas2017cellular,Houghteling_2015}.

Parallel met de stijging van de incidentie is het dieet van de zuigeling veranderd. Borstvoeding was in het verleden gebruikelijker \cite{world1981contemporary,Victora_2016,Rollins_2016}. Het dieet is verschoven naar flesvoeding of vervroegd vaste voeding. In tegenstelling tot de twee laatstgenoemde diëten verschaft borstvoeding een positief modulerend werking op het immuunsysteem van de zuigeling.

De ontwikkeling van astma is te verwijten aan een verlaagde verhouding van T\textsubscript{h}1/T\textsubscript{h}2 cellen en een gebrek aan Tregs.

\begin{figure}
    \centering
    \begin{tikzpicture}[ultra thick,auto,blok/.style={draw,thick,fill=blue!25}]
        \foreach \x in {0,120,240} {
        \Receptor{\x}
    }
        \cellmembrane{blue}{a}
        \nucleus{blue}{b}
        \node[left] at (a.north west) {T\textsubscript{h}1 cel};
    \begin{scope}[xshift=18em]
    \foreach \x in {0,120,240} {
    \Receptor{\x}
}
    \cellmembrane{red}{z}
    \nucleus{red}{b}
    \node[right] at (z.north east) {T\textsubscript{h}2 cel};
\end{scope}

    \begin{scope}[xshift=9em,yshift=9em]
    \foreach \x in {0,120,240} {
    \Receptor{\x}
}
    \cellmembrane{green}{r}
    \nucleus{green}{l}
    \node[right] at (r.north east) {Treg cel};
    \end{scope}
    \node[yshift=10em,blok] (tgfb) at (r) {TGF-$\beta$};
    \node[above of=tgfb,blok] (microbioom) {Divers microbioom};
    \node[above of=microbioom,blok] (lkvoz) {\glspl{lovk}};
    \node[blok,fill=red!25] (atopie) at (9em,-10em) {Atopie};
                    \begin{scope}[thick]
                        \draw[<->] (a) -- (z) node[midway] {differentiatie};
                        \draw[-|] (r.east) to[out=0,in=90] node{$\downarrow$ differentiatie} ($(z)+(0,2)$);
                        \draw[->] (r.west) to[out=180,in=90] node[anchor=south east]{$\uparrow$ differentiatie} ($(a)+(0,2)$);
                        \draw[-|] (a) to[bend right] node[anchor=north east] {vermindert}(atopie);
                        \draw[->] (z) to[bend left] node[auto]{verhoogt} (atopie);
                        \path[->] (lkvoz) edge (microbioom)
                                  (microbioom) edge (tgfb)
                                  (tgfb) edge ($(r)+(0,2)$)
                                  ;
                \end{scope}
    \end{tikzpicture}
    \captionbelow{Hypothetische werkingsmechanisme van borstvoeding op vermindering van atopie. \Glspl{lovk} verhoogt de diversiteit van het microbioom. Deze verhoging leidt tot een productieverhoging van de cytokine TGF-$\beta$. TGF-$\beta$ bevordert de groei van Tregs. Tregs hebben een immuunonderdrukkende functie en zorgen voor een hogere $\frac{\text{T\textsubscript{h}1}}{\text{T\textsubscript{h}2}}$. Tezamen veroorzaken zij een verlaagde predispositie op ontwikkeling van astma.}
    \label{fig:atopie_ontwikkeling}
\end{figure}


\section{Methode}
    \subsection{Zoek \& selectie}

    \subsection{Standaardisering}

    %\subsection{Kwaliteitsbeoordeling}
\section{resultaten}

\section{Discussie \& conclusie}

\printbibliography 
\end{document}
