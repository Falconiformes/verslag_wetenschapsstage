\documentclass[table,abstract=true]{scrartcl}
\usepackage{lmodern}
\usepackage{microtype}
\usepackage[dutch]{babel}
\usepackage[utf8]{inputenc}
\usepackage{csquotes}
%%% Font & taal
\usepackage{booktabs,array,multirow,tabularx} %% Tabellen libraries
\usepackage{siunitx}
\usepackage[version=4]{mhchem}
%%% Figuur programmeren
\usepackage{tikz}
\usetikzlibrary{positioning,calc,intersections,matrix}
\newcommand{\Receptor}[1]{%
\begin{scope}[rotate around={#1:(0,0)}]
    \draw[%
    ultra thick,
    smooth,
    line cap=round
    ] (-.50em,1em) -- ++(0,2.5em) 
    to[bend left] ++(-.5em,1.25em) to[bend left] ++ (-1.0em,1.25em);
%
    \draw[%
    xscale=-1,
    ultra thick,
    smooth,
    line cap=round
    ] (-.50em,1em) -- ++(0,2.5em) 
    to[bend left] ++(-.5em,1.25em) to[bend left] ++ (-1.0em,1.25em);
\end{scope}
}
\newcommand{\cellmembrane}[2]{%
        \node[fill=#1!45,
        minimum size=5em,
        inner sep=0,
        draw,
        circle,ultra thick
        ] (#2) at (0,0) {};
}
\newcommand{\nucleus}[2]{%
        \node[%
        shading=ball,
        ball color=#1!50,
        minimum size=2.5em,
        inner sep=0,
        draw,
        circle,
        ultra thick
        ] (#2) at (-0.20,0.20) {};
}
\newcommand{\flowblock}[4][]{%
    \node[%
        inner sep=5,
        minimum width=4cm,
        draw,
        fill=blue!25,
        align=center,
        rounded corners,
        #1
                     ]
                     (#2) {#3\\ ($n = #4$)};
}
\tikzset{>=latex}
\usepackage{graphicx}
%%% Marges bepaling
\usepackage{biblatex}
\addbibresource{Bronnen/bib.bib}
%%% Bibliografie
\usepackage[pdfusetitle]{hyperref}
\hypersetup{%
    colorlinks=true,
    citecolor=blue,
    pdfauthor={Edon Namani; Martijn Schuiling; Gijs Stuart; Casper Jansen; Martein Leen},
    pdftitle={Effect van borstvoeding op ontwikkelingskans van astma},
    pdfsubject={Immunologie; Allergie},
    pdfkeywords={Immunologie; Allergie; Borstvoeding; Astma; Odds ratio}
}
\usepackage[dutch]{cleveref}
%%% Hyperlinks & easy referencing
\usepackage[acronym,smallcaps]{glossaries}
\usepackage{scrlayer-scrpage}
\ihead{Borstvoeding \& astma}
\ifoot{\copyright{} 2019 \LaTeXe{} foundations}
\usepackage{threeparttable}
\addtokomafont{captionlabel}{\bfseries}
\KOMAoptions{DIV=calc}
%%%% acronyms
\newacronym{lovk}{lovk}{lange onverzadigde vetzuurketen}
\makeglossaries

\title{Effect van borstvoeding op ontwikkeling van astma}
\subtitle{Wetenschapsstage}
\author{Edon Namani\thanks{Praeses} \and Martijn Schuiling\thanks{Co\"ordinator} \and Gijs Stuart \and Casper Jansen \and Martein Leen}
\date{\today}
\subject{Immunologie \& allergie}
\begin{document}
\maketitle
    \begin{abstract}
        De ontwikkeling van astma is een multifactorieel proces. Het type dieet van een zuigeling is een belangrijke factor in de vroege sensibilisering en de ontwikkeling van T\textsubscript{h}2-cel respons. In afgelopen decennia is flesvoeding gebruikelijker geworden dan borstvoeding. Onbekend is of de toegenomen incidentie van astma te wijten is aan de dieetverandering van de zuigeling. 
        
        Dit onderzoek toont aan dat borstvoeding significant de ontwikkeling van astma vermindert bij zuigelingen. In een literatuuronderzoek zijn na een selectie 3 cohort studies en 1 RCT met elkaar vergeleken. Elk onderzoek vermeldde een $OR > 1$ en $P < 0,05$. Derhalve moet verder onderzoek in het beschermende fysiologische mechanisme van borstvoeding uitgevoerd worden.
    \end{abstract}

\section{Introductie}
In de afgelopen decennia is de incidentie van astma en andere type I hypersensitiviteit morbi ferm toegenomen \cite{Platts_Mills_2015}. Deze toegenomen incidentie veroorzaakt een verhoogde zorglast voor de maatschappij. Astma is een kinderlijke chronische respiratoire morbus. Meerdere factoren dragen bij aan de ontwikkeling van astma. De factoren zijn de genetische predispositie, de wijze van bevalling en het dieet \cite{abbas2017cellular,Houghteling_2015}.

De factoren leiden tot ongewenste sensibilisering van T-cellen voor onschadelijke antigenen en differentiatie van T\textsubscript{h}2-cellen. T\textsubscript{h}2-cellen zijn verantwoordelijk voor IgE isotype switching door plasmacellen. IgE~antilichamen doen mastocyten activeren, als ze weder in contact komen met het antigen. Activatie van de mastocyten in de context van astma betekent de uitscheiding van leukotriënen, wat zorgt voor de vernauwing van de luchtwegen. Op deze wijze zorgt een T\textsubscript{h}2-respons voor dyspneu bij astmapatiënten.

Parallel met de stijging van de incidentie is het dieet van de zuigeling veranderd. Borstvoeding was in het verleden gebruikelijker \cite{world1981contemporary,Victora_2016,Rollins_2016}. Het dieet is verschoven naar flesvoeding of vervroegd vaste voeding. Daarom is onderzocht wat het effect van borstvoeding is (in vergelijking met flesvoeding) bij zuigelingen op de ontwikkeling van astma tot het zesde levensjaar.


\section{Methode}
    \subsection{Zoekstrategie \& selectie}
    Om te onderzoeken wat het effect van borstvoeding is bij zuigelingen op de ontwikkeling van astma tot het zesde levensjaar in vergelijking met flesvoeding, is er in dit onderzoek gebruik gemaakt van het Pubmed-archief. Na een uitgebreide selectie zijn er vier artikelen uitgezocht die op deze onderzoeksvraag een antwoord kunnen geven.\hfil

    De zoekstrategie maakte gebruik van de \textsc{All}-database en de \textsc{MeSH}-database. De zoektermen in beide databases waren \emph{incidence}, \emph{breast feeding} en \emph{asthma}. Hierbij was elke zoekterm met elkaar verbonden door het voegwoord ``\uppercase{and}''.

Deze zoekopdracht leverde 656 artikelen op (zie \cref{fig:flowchart_artcl_selectie}). Na exclusie van reviews, artikelen zonder een abstract en artikelen die geen betrekking hadden op mensen bleven er 58 RCT \& cohort studies over. Deze werden nogmaals door een selectieprocedure gehaald, waarbij artikelen met een ongerelateerde titel en een ongerelateerde abstract weg werden gefilterd ($n = 48$).
Dit liet tien artikelen over, waarvan vier artikelen het best rekening hielden met de confounding variabelen. Deze vier artikelen werden opgenomen in het onderzoek \cite{chandra1997five,den_Dekker_2016,oddy1999association,dell2001breastfeeding}.

\begin{figure}
    \centering
    \begin{tikzpicture}[thick]
        \matrix [row sep=2em,column sep=3em]
        {
            \flowblock{mesh_data}{Artikels in \textsc{MeSH}-database}{25} &\flowblock{all_data}{Artikels in \textsc{All}-database}{631} \\
                                                                          &\flowblock[fill=red!25]{lol}{\parbox{4.2cm}{%
                                                                                  \begin{itemize}   
                                                                                      \item Geen mens
                                                                                      \item Review
                                                                                      \item Geen abstract
                                                                                    \end{itemize}
                                                                            }}{598}\\
            \flowblock{cohort_clinical}{RCT \& Cohort studies}{58} & \\
                                                                   & \flowblock[fill=red!25]{artcl_unuse}{\parbox{4.4cm}{%
                                                                        \begin{itemize}
                                                                            \item Ongerelateerde titel
                                                                            \item Ongerelateerde abstract
                                                                        \end{itemize}
                                                                    }
                                                                   }
                                                                   {48}\\
            \flowblock{Bruikbare Artikelen}{Bruikbare artikelen}{10} & \\
                            & \\
            \flowblock{artcl_best}{Meest geschikte artikelen}{4} & \\
        };
        \path[->]
            (mesh_data) edge (cohort_clinical)
            (cohort_clinical) edge (Bruikbare Artikelen)
            (Bruikbare Artikelen) edge node[auto,text width=5cm] {Rekening houdend met confounding variabelen}(artcl_best)
            ($(mesh_data)!.5!(cohort_clinical)$) edge (lol)
            ($(cohort_clinical)!.5!(Bruikbare Artikelen)$) edge (artcl_unuse)
            ;
        \draw (mesh_data) edge (all_data);
    \end{tikzpicture}
    \captionbelow{Het filterproces omvatte de selectie op cohort studies \& RCT, overeenstemming van de inhoud van het artikel met onze vraagstelling en betrekking van confounding variabelen.}
    \label{fig:flowchart_artcl_selectie}
\end{figure}


\section{Resultaten}
\begin{table}
   \centering
   \captionabove{Overzicht van 4 studies naar de incidentie van astma bij zuigelingen met borstvoeding en met flesvoeding.}
   \begin{threeparttable}
   \resizebox{\textwidth}{!}{% 
       \begin{tabular}{ll*{2}{S[table-format=4.0]}cclc}
        \toprule
        Artikel & Nationaliteit & \multicolumn{2}{c}{$n_{tot}$} & Follow-up    & Borstvoeding & Onderzoeksvorm & $P$-waarde \\
        \cmidrule(lr){3-4}
                &               & {$n_{borst}$}                 & {$n_{fles}$} & (jaar)       & (maand)        &             & $< 0,05$ \\
        \midrule
        Oddy1999    & Australië & 1094 & 873  & 6          & 4 & Cohort studie & Waar \\
        Dekker2016  & Nederland & 4534 & 1141 & 6          & 2 & Cohort studie & Waar \\
        Dell2001    & Canada    & 369  & 1815 & n.v.t.\tnote{1} & 9 & Cohort studie & Waar \\
        Chandra1997 & Canada    & 216  & 72   & 5          & 6 & RCT           & Waar \\
        \bottomrule
    \end{tabular}
}
\begin{tablenotes}
\item[1] Onduidelijke vermelding van follow-up duur;
\end{tablenotes}

\end{threeparttable}
\label{tab:overzicht_artcls}
\end{table}

Voor het onderzoek zijn dus 4 artikelen geselecteerd. 3 hiervan zijn een cohort studie en 1 is een RCT. In alle 4 de artikelen wordt het effect van borstvoeding op astma onderzocht. Dit wordt gedaan door een hoeveelheid kinderen borstvoeding te geven en een bepaalde hoeveelheid kinderen flesvoeding te geven. In de RCT is random geselecteerd wie borstvoeding gaat krijgen en wie niet. In de anderen 3 studies is een groep mensen bijgehouden die zelf bepalen of er borstvoeding wordt gegeven of niet. Deze borstvoeding wordt per onderzoek voor verschillende periodes gegeven. Vervolgens wordt voor een periode van ongeveer 6 jaar bijgehouden of de kinderen astma krijgen. Uiteindelijk is vergeleken of de kinderen met borstvoeding minder vaak astma kregen dan de kinderen met flesvoeding. Bij deze onderzoeken is er ook nog rekening gehouden met een aantal factoren die de incidentie van astma zouden kunnen beïnvloeden. Deze zijn aangegeven in \cref{tab:confounding_var}.

De resultaten van dit onderzoek zijn verwerkt in \cref{tab:overzicht_artcls}. Hierin zijn de belangrijkste gegevens verwerkt waarmee de verschillende artikelen makkelijk vergeleken kunnen worden. Ook valt het resultaat van de verschillende onderzoeken in deze tabel te zien en is makkelijk te zien of dit resultaat significant is. 

In de laatste rij van de tabel staan de $P$-waardes uit de artikelen en of ze groter of kleiner zijn dan $0,05$. Wanneer de waarde uit een artikel kleiner is dan $0,05$ betekent dit dat het resultaat uit dit onderzoek significant is (in \cref{tab:overzicht_artcls} aangegeven met “waar”). Dit betekent dat volgens dit onderzoek, borstvoeding een significant effect heeft op het verminderen van de incidentie van astma tot een bepaald levensjaar. Zoals te zien is, hebben alle vier onderzoeken een significant resultaat. Alle vier laten dus zien dat astma minder voor komt binnen een bepaalde bevolkingsgroep wanner borstvoeding wordt gegeven in plaats van flesvoeding. 

Het is natuurlijk nog wel van belang om te bekijken of de onderzoeken wel met elkaar vergeleken kunnen worden en niet op verschillende manieren uitgevoerd zijn. De duur van de follow-up en de periode tot wanneer borstvoeding is gegeven verschilt per onderzoek en staan ook weergeven in de tabel . Bij het onderzoek is  gekeken naar de incidentie van astma tot het zesde levensjaar. De verschillende follow-ups van de onderzoeken zijn dan ook 6 jaar of liggen hier heel dicht bij. 

Ook is vergeleken uit welke landen de verschillende mensen uit de onderzoeken komen. Per land kan de incidentie van astma verschillen, maar ook de effecten van moedermelk kunnen per land anders zijn. 

Als laatste is er gekeken naar de hoeveelheid mensen die het onderzoek heeft gebruikt. Hierin is niet alleen naar het totaal aantal mensen gekeken, maar ook naar hoeveel kinderen specifiek borstvoeding hebben gehad en hoeveel kinderen flesvoeding hebben gehad. 

\begin{table}[h]
    \centering
    \captionabove{Overzicht van de confounding variabelen waarmee de 4 artikelen rekening meegehouden hebben. Risicogroep betreft zuigelingen waarbij in de familie gevallen zijn met allergieën.}
    \begin{tabular}{l*4c}
        \toprule
        Confounding factors & Oddy1999 & Dekker2016 & Dell2001 & Chandra1997 \\
        \midrule
        Sekse      & \good & \good & \good & \good \\
Huisdieren         & \bad      & \good & \bad  &\good       \\
Economische status & \good      & \good & \good &\good       \\
Roken familie      & \good      & \good & \good &       \\
BMI moeder         & \bad  & \good & \bad  & \bad  \\
Leeftijd moeder    &  \bad     & \good & \bad  &       \\
Etniciteit         & \good & \good & \bad  & \bad  \\
Geboortegewicht    & \bad  & \good & \good & \good \\
Zwangerschapsduur  & \good & \good & \good & \bad  \\
Risicogroep        & \bad  & \bad  & \bad  & \good \\
        \bottomrule
    \end{tabular}
    \label{tab:confounding_var}
\end{table}
\section{Discussie \& conclusie}
In deze studie is er gekeken naar het effect van borstvoeding op de incidentie van astma tot het zesde levensjaar. In de vier studies die in dit onderzoek zijn onderzocht, bleek in alle gevallen, dat borstvoeding een significant effect heeft op het verminderen van astma voor een bepaald levensjaar. 
D.m.v. een filterproces is er uit een groot aantal artikelen, de vier meest geschikte artikelen gekozen, om zo betrouwbaar mogelijke resultaten te vinden. Daarnaast laten de $P$-waarden van de vier onderzoeken zien, dat borstvoeding een significant effect heeft op het verminderen van incidentie. 
Wel moet er vermeld worden dat de studie door Dell 2001 niet volledig was (nl. het ontbreken van de follow-up periode). Desondanks gaf ook deze studie een significant effect maar is wellicht minder relevant indien de follow-up op een leeftijd ouder dan zes jaar is gedaan. 

Concluderend heeft borstvoeding een positief effect op het voorkomen van astma. Borstvoeding zal dus, indien mogelijk, gekozen moeten worden over flesvoeding. Echter hangt het ontwikkelen van astma af van vele factoren, en geeft borstvoeding geen garantie. 


Kijkend naar de onderzochte onderzoeken is er te zien, dat er bij alle vier de artikelen er een significante verandering is in de ontwikkeling van astma ten gevolgen van het drinken van borstvoeding. Er is hier gekeken naar borstvoeding tot de leeftijd van 6 maanden. Ook is er gekeken naar de verschillende bevolkingsgroepen. Hier was in de vier gevonden artikelen geen groot verschil tussen. Het is dus aannemelijk dat kinderen die tot 6 maanden borstvoeding krijgen, een minder grote kans hebben op het ontwikkelen van astma.


%%% Vervolgonderzoek - mechanisme van borstvoeding
%%% Maak een link tussen epidemiologische onderzoek naar een fysiologische mechanischistische onderzoming als borstvoedin.
\subsection{Pathofysiologie}
Hier is ook een duidelijke fysiologische verklaring voor.
De ontwikkeling van astma is te wijten aan een verlaagde verhouding van T\textsubscript{h}1/T\textsubscript{h}2 cellen en een gebrek aan Tregs (zie \cref{fig:atopie_ontwikkeling}). CD4\textsuperscript{+} T~cellen worden opgedeeld in deelverzamelingen op basis van de cytokines die ze uitscheiden. T\textsubscript{h}1~cellen scheiden IFN-$\gamma$ uit, T\textsubscript{h}2~cellen IL-5 en Tregs IL-10. Door de uitscheiding van verschillende cytokines heeft elke deelverzameling een eigen unieke effectorfunctie. T\textsubscript{h}1~cellen zorgen voor de doding van intracellulaire pathogenen, T\textsubscript{h}2~cellen voor doding van parasieten en Tregs voor onderdrukking van het immuunsysteem. Het doden van parasieten gebeurt door de activatie van eosinofielen, mestcellen en door immunoglobuline klasseswitching naar IgE. Hetzelfde fysiologische mechanisme ligt ook ten grondslag aan de pathologie van astma.

Moederlijk melk bevat \glspl{lovk} \cite{das2002essential}. \Glspl{lovk} verbeteren de diversiteit van het darmflora. De diversiteit zorgt voor een verhoogde productie van TGF-$\beta$ \cite{das2002essential,Das_2004}. Deze cytokine is een mitogene factor voor Tregs. Tregs respectievelijk bevorderen en remmen de differentiatie van T\textsubscript{h}1~cellen en van T\textsubscript{h}2-cellen \cite{penttila2010milk}. Verder remmen de Tregs de productie van ontsteking veroorzakende cytokinen. Door verhoging van T\textsubscript{h}1/T\textsubscript{h}2 ratio en remming van de ontsteking heeft moederlijk melk een atopie verminderend effect.

\subsection{Verder onderzoek}
Deze literaire epidemiologische studie valideerde het bestaande fysiologisch mechanisme van borstvoeding. Daarom is aangeraden verder onderzoek uit te voeren in het mechanisme van verhoogde TGF-$\beta$ productie en kunstmatige productie van de \glspl{lovk}. Deze onderzoeken beogen het ontwikkelen van flesvoeding met dezelfde bescherming als borstvoeding.
\begin{figure}
    \centering
    \begin{tikzpicture}[ultra thick,auto,blok/.style={draw,thick,fill=blue!25}]
        \foreach \x in {0,120,240} {
        \Receptor{\x}
    }
        \cellmembrane{blue}{a}
        \nucleus{blue}{b}
        \node[left] at (a.north west) {T\textsubscript{h}1 cel};
    \begin{scope}[xshift=18em]
    \foreach \x in {0,120,240} {
    \Receptor{\x}
}
    \cellmembrane{red}{z}
    \nucleus{red}{b}
    \node[right] at (z.north east) {T\textsubscript{h}2 cel};
\end{scope}

    \begin{scope}[xshift=9em,yshift=9em]
    \foreach \x in {0,120,240} {
    \Receptor{\x}
}
    \cellmembrane{green}{r}
    \nucleus{green}{l}
    \node[right] at (r.north east) {Treg cel};
    \end{scope}
    \node[yshift=10em,blok] (tgfb) at (r) {TGF-$\beta$};
    \node[above of=tgfb,blok] (microbioom) {Divers microbioom};
    \node[above of=microbioom,blok] (lkvoz) {\glspl{lovk}};
    \node[blok,fill=red!25] (atopie) at (9em,-10em) {Atopie};
                    \begin{scope}[thick]
                        \draw[<->] (a) -- (z) node[midway] {differentiatie};
                        \draw[-|] (r.east) to[out=0,in=90] node{$\downarrow$ differentiatie} ($(z)+(0,2)$);
                        \draw[->] (r.west) to[out=180,in=90] node[anchor=south east]{$\uparrow$ differentiatie} ($(a)+(0,2)$);
                        \draw[-|] (a) to[bend right] node[anchor=north east] {vermindert}(atopie);
                        \draw[->] (z) to[bend left] node[auto]{verhoogt} (atopie);
                        \path[->] (lkvoz) edge (microbioom)
                                  (microbioom) edge (tgfb)
                                  (tgfb) edge ($(r)+(0,2)$)
                                  ;
                \end{scope}
    \end{tikzpicture}
    \captionbelow{Hypothetische werkingsmechanisme van borstvoeding op vermindering van atopie. \Glspl{lovk} verhoogt de diversiteit van het microbioom. Deze verhoging leidt tot een productieverhoging van de cytokine TGF-$\beta$. TGF-$\beta$ bevordert de groei van Tregs. Tregs hebben een immuunonderdrukkende functie en zorgen voor een hogere $\frac{\text{T\textsubscript{h}1}}{\text{T\textsubscript{h}2}}$. Tezamen veroorzaken zij een verlaagde predispositie op ontwikkeling van astma.}
    \label{fig:atopie_ontwikkeling}
\end{figure}

\printglossaries
\newpage
\printbibliography 
\end{document}
