\documentclass[abstract=true]{scrartcl}
\usepackage{lmodern}
\usepackage{microtype}
\usepackage[dutch]{babel}
\usepackage[utf8]{inputenc}
\usepackage{csquotes}
%%% Font & taal
\usepackage{booktabs,array,multirow,tabularx} %% Tabellen libraries
\usepackage{siunitx}
\usepackage[version=4]{mhchem}
%%% Figuur programmeren
\usepackage{tikz}
\usetikzlibrary{positioning,calc,intersections,matrix}
\newcommand{\Receptor}[1]{%
\begin{scope}[rotate around={#1:(0,0)}]
    \draw[%
    ultra thick,
    smooth,
    line cap=round
    ] (-.50em,1em) -- ++(0,2.5em) 
    to[bend left] ++(-.5em,1.25em) to[bend left] ++ (-1.0em,1.25em);
%
    \draw[%
    xscale=-1,
    ultra thick,
    smooth,
    line cap=round
    ] (-.50em,1em) -- ++(0,2.5em) 
    to[bend left] ++(-.5em,1.25em) to[bend left] ++ (-1.0em,1.25em);
\end{scope}
}
\newcommand{\cellmembrane}[2]{%
        \node[fill=#1!45,
        minimum size=5em,
        inner sep=0,
        draw,
        circle,ultra thick
        ] (#2) at (0,0) {};
}
\newcommand{\nucleus}[2]{%
        \node[%
        shading=ball,
        ball color=#1!50,
        minimum size=2.5em,
        inner sep=0,
        draw,
        circle,
        ultra thick
        ] (#2) at (-0.20,0.20) {};
}
\newcommand{\flowblock}[4][]{%
    \node[%
        inner sep=5,
        minimum width=4cm,
        draw,
        fill=blue!25,
        align=center,
        rounded corners,
        #1
                     ]
                     (#2) {#3\\ ($n = #4$)};
}
\tikzset{>=latex}
\usepackage{graphicx}
%%% Marges bepaling
\usepackage{biblatex}
\addbibresource{Bronnen/bib.bib}
%%% Bibliografie
\usepackage[pdfusetitle]{hyperref}
\hypersetup{%
    colorlinks=true,
    citecolor=blue,
    pdfauthor={Edon Namani; Martijn Schuiling; Gijs Stuart; Casper Jansen; Martein Leen},
    pdftitle={Effect van borstvoeding op ontwikkelingskans van astma},
    pdfsubject={Immunologie; Allergie},
    pdfkeywords={Immunologie; Allergie; Borstvoeding; Astma; Odds ratio}
}
\usepackage[dutch]{cleveref}
%%% Hyperlinks & easy referencing
\usepackage[acronym,smallcaps]{glossaries}
\usepackage{scrlayer-scrpage}
\ihead{Borstvoeding \& astma}
\ifoot{\copyright{} 2019 \LaTeXe{} foundations}
\usepackage{threeparttable}
\addtokomafont{captionlabel}{\bfseries}
\KOMAoptions{DIV=calc}
%%%% acronyms
\newacronym{lovk}{lovk}{lange onverzadigde vetzuurketen}
\makeglossaries

\title{Effect van borstvoeding op ontwikkelingskans van astma}
\subtitle{Wetenschapsstage}
\author{Edon Namani\thanks{Praeses} \and Martijn Schuiling\thanks{Co\"ordinator} \and Gijs Stuart \and Casper Jansen \and Martein Leen}
\date{\today}
\subject{Immunologie \& allergie}
\begin{document}
\maketitle
    \begin{abstract}

    \end{abstract}

\section{Introductie}
In de afgelopen decennia is de incidentie van astma en andere type I hypersensitiviteit morbi ferm toegenomen \cite{Platts_Mills_2015}. Deze toegenomen incidentie veroorzaakt een verhoogde zorglast voor de maatschappij. Astma is een kinderlijke chronische respiratoire morbus. Meerdere factoren dragen bij aan de ontwikkeling van astma. De factoren zijn de genetische predispositie, de wijze van bevalling en het dieet \cite{abbas2017cellular,Houghteling_2015}.

Parallel met de stijging van de incidentie is het dieet van de zuigeling veranderd. Borstvoeding was in het verleden gebruikelijker \cite{world1981contemporary,Victora_2016,Rollins_2016}. Het dieet is verschoven naar flesvoeding of vervroegd vaste voeding. Daarom hebben wij onderzocht wat het effect van borstvoeding is bij zuigelingen op de ontwikkelingskans van astma voor de zesde levensjaar in vergelijking tot flesvoeding.


\section{Methode}
    \subsection{Zoek \& selectie}
\begin{figure}
    \centering
    \begin{tikzpicture}[thick]
        \matrix [row sep=2em,column sep=3em]
        {
            \flowblock{mesh_data}{Artikels in \textsc{MeSH}-database}{25} &\flowblock{all_data}{Artikels in \textsc{All}-database}{631} \\
                                                                          &\flowblock[fill=red!25]{lol}{\parbox{4.2cm}{%
                                                                                  \begin{itemize}   
                                                                                      \item Geen mens
                                                                                      \item Review
                                                                                      \item Geen abstract
                                                                                    \end{itemize}
                                                                            }}{598}\\
            \flowblock{cohort_clinical}{RCT \& Cohort studies}{58} & \\
                                                                   & \flowblock[fill=red!25]{artcl_unuse}{\parbox{4.4cm}{%
                                                                        \begin{itemize}
                                                                            \item Ongerelateerde titel
                                                                            \item Ongerelateerde abstract
                                                                        \end{itemize}
                                                                    }
                                                                   }
                                                                   {48}\\
            \flowblock{Bruikbare Artikelen}{Bruikbare artikelen}{10} & \\
                            & \\
            \flowblock{artcl_best}{Meest geschikte artikelen}{4} & \\
        };
        \path[->]
            (mesh_data) edge (cohort_clinical)
            (cohort_clinical) edge (Bruikbare Artikelen)
            (Bruikbare Artikelen) edge node[auto,text width=5cm] {Rekening houdend met confounding variabelen}(artcl_best)
            ($(mesh_data)!.5!(cohort_clinical)$) edge (lol)
            ($(cohort_clinical)!.5!(Bruikbare Artikelen)$) edge (artcl_unuse)
            ;
        \draw (mesh_data) edge (all_data);
    \end{tikzpicture}
    \captionbelow{Het filterproces omvatte de selectie op cohort studies \& RCT, overeenstemming van de inhoud van het artikel met onze vraagstelling en betrekking van confounding variabelen.}
    \label{fig:flowchart_artcl_selectie}
\end{figure}

    \subsection{Standaardisering}

    %\subsection{Kwaliteitsbeoordeling}
\section{resultaten}
\begin{table}
   \centering
   \captionabove{Verzin nog iets}
   \begin{threeparttable}
   \resizebox{\textwidth}{!}{% 
       \begin{tabular}{ll*{2}{S[table-format=4.0]}cclc}
        \toprule
        Artikel & Nationaliteit & \multicolumn{2}{c}{$n_{tot}$} & Follow-up    & Borstvoeding & Onderzoeksvorm & $P$-waarde \\
        \cmidrule(lr){3-4}
                &               & {$n_{borst}$}                 & {$n_{fles}$} & (jaar)       & (maand)        &             & $< 0,05$ \\
        \midrule
        Oddy1999    & Australië & 1094 & 873  & 6          & 4 & Cohort studie & Waar \\
        Dekker2016  & Nederland & 4534 & 1141 & 6          & 2 & Cohort studie & Waar \\
        Dell2001    & Canada    & 369  & 1815 & *\tnote{1} & 9 & Cohort studie & Waar \\
        Chandra1996 & Canada    & 216  & 72   & 5          & 6 & RCT           & Waar \\
        \bottomrule
    \end{tabular}
}
\begin{tablenotes}
\item[1]Gwn weglaten;
\end{tablenotes}

\end{threeparttable}
\label{tab:overzicht_artcls}
\end{table}

De resultaten van het onderzoek zijn verwerkt in \cref{tab:overzicht_artcls}. Hierin zijn de belangrijkste gegevens verwerkt waar de verschillende artikelen die wij hebben gebruikt makkelijk vergeleken kunnen worden. Ook valt het resultaat van de verschillende onderzoeken in deze tabel te zien en is makkelijk te zien of dit resultaat significant is. In de laatste rij van de tabel hebben we staan of de $P$-waardes uit de artikelen groter of kleiner zijn dan $0,05$. Wanneer de waarde uit een artikel kleiner is dan $0,05$ betekent dat het resultaat uit dit onderzoek significant is (in \cref{tab:overzicht_artcls} aangegeven met “waar”). Dit betekent dat volgens dit onderzoek, borstvoeding een significant effect heeft op het verminderen van astma voor een bepaald levensjaar. Zoals te zien is, hebben alle 4 onderzoeken die wij hebben gebruikt een significant resultaat. Alle 4 de onderzoeken laten zien dat borstvoeding een belangrijk effect op de incidentie van Astma heeft. Het is natuurlijk nog wel van belang te kijken naar of de onderzoeken wel met elkaar vergeleken kunnen worden en niet op verschillende manieren gedaan zijn. Het levensjaar van de follow-up en de periode tot wanneer borstvoeding is gegeven verschilt ook per onderzoek en staat ook in de tabel weergeven. Ook hebben we vergeleken uit welke landen de verschillende mensen uit de onderzoeken komen. per land kan de incidentie van Astma verschillen, maar ook de effecten van moedermelk kunnen per land anders zijn. Als laatste is er gekeken naar hoeveel mensen het onderzoek heeft gebruikt. Hierin is niet alleen naar het totaal aantal mensen gekeken, maar ook naar hoeveel kinderen specifiek borstvoeding hebben gehad en hoeveel kinderen flesvoeding hebben gehad. 

\section{Discussie \& conclusie}

%%% Vervolgonderzoek - mechanisme van borstvoeding
De ontwikkeling van astma is te verwijten aan een verlaagde verhouding van T\textsubscript{h}1/T\textsubscript{h}2 cellen en een gebrek aan Tregs (zie \cref{fig:atopie_ontwikkeling}). CD4\textsuperscript{+} T~cellen worden opgedeeld in deelverzamelingen op basis van de cytokines die ze uitscheiden. T\textsubscript{h}1~cellen scheiden IFN-$\gamma$ uit, T\textsubscript{h}2~cellen IL-5 en Tregs IL-10. Door de uitscheiding van verschillende cytokines heeft elke deelverzameling een eigen unieke effector functie. T\textsubscript{h}1~cellen zorgen voor de doding van intracellulaire pathogenen, T\textsubscript{h}2~cellen voor doding van parasieten en Tregs onderdrukking van het immuunsysteem. Het doden van parasieten gebeurt door de activatie van eosinofielen, mest cellen en door immunoglobuline klasse switching naar IgE. Hetzelfde fysiologische mechanisme ligt ook ten grondslag bij de pathologie van astma.

Moederlijk melk bevat \glspl{lovk} \cite{das2002essential}. \Glspl{lovk} verbeteren de diversiteit van het darmflora. De diversiteit zorgt voor een verhoogde productie van TGF-$\beta$ \cite{das2002essential,Das_2004}. Deze cytokine is een mitogene factor voor Tregs. Tregs bevorderen de differentiatie van T\textsubscript{h}1~cellen en remt van T\textsubscript{h}2-cellen \cite{penttila2010milk}. Verder remmen de Tregs de productie van ontsteking veroorzakende cytokinen. 

\begin{figure}
    \centering
    \begin{tikzpicture}[ultra thick,auto,blok/.style={draw,thick,fill=blue!25}]
        \foreach \x in {0,120,240} {
        \Receptor{\x}
    }
        \cellmembrane{blue}{a}
        \nucleus{blue}{b}
        \node[left] at (a.north west) {T\textsubscript{h}1 cel};
    \begin{scope}[xshift=18em]
    \foreach \x in {0,120,240} {
    \Receptor{\x}
}
    \cellmembrane{red}{z}
    \nucleus{red}{b}
    \node[right] at (z.north east) {T\textsubscript{h}2 cel};
\end{scope}

    \begin{scope}[xshift=9em,yshift=9em]
    \foreach \x in {0,120,240} {
    \Receptor{\x}
}
    \cellmembrane{green}{r}
    \nucleus{green}{l}
    \node[right] at (r.north east) {Treg cel};
    \end{scope}
    \node[yshift=10em,blok] (tgfb) at (r) {TGF-$\beta$};
    \node[above of=tgfb,blok] (microbioom) {Divers microbioom};
    \node[above of=microbioom,blok] (lkvoz) {\glspl{lovk}};
    \node[blok,fill=red!25] (atopie) at (9em,-10em) {Atopie};
                    \begin{scope}[thick]
                        \draw[<->] (a) -- (z) node[midway] {differentiatie};
                        \draw[-|] (r.east) to[out=0,in=90] node{$\downarrow$ differentiatie} ($(z)+(0,2)$);
                        \draw[->] (r.west) to[out=180,in=90] node[anchor=south east]{$\uparrow$ differentiatie} ($(a)+(0,2)$);
                        \draw[-|] (a) to[bend right] node[anchor=north east] {vermindert}(atopie);
                        \draw[->] (z) to[bend left] node[auto]{verhoogt} (atopie);
                        \path[->] (lkvoz) edge (microbioom)
                                  (microbioom) edge (tgfb)
                                  (tgfb) edge ($(r)+(0,2)$)
                                  ;
                \end{scope}
    \end{tikzpicture}
    \captionbelow{Hypothetische werkingsmechanisme van borstvoeding op vermindering van atopie. \Glspl{lovk} verhoogt de diversiteit van het microbioom. Deze verhoging leidt tot een productieverhoging van de cytokine TGF-$\beta$. TGF-$\beta$ bevordert de groei van Tregs. Tregs hebben een immuunonderdrukkende functie en zorgen voor een hogere $\frac{\text{T\textsubscript{h}1}}{\text{T\textsubscript{h}2}}$. Tezamen veroorzaken zij een verlaagde predispositie op ontwikkeling van astma.}
    \label{fig:atopie_ontwikkeling}
\end{figure}

\printglossaries
\printbibliography 
\end{document}
