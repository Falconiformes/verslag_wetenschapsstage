\documentclass[abstract=true]{scrartcl}
\usepackage{lmodern}
\usepackage{microtype}
\usepackage[dutch]{babel}
\usepackage[utf8]{inputenc}
\usepackage{csquotes}
%%% Font & taal
\usepackage{booktabs,array,multirow,tabularx} %% Tabellen libraries
\usepackage{siunitx}
\usepackage[version=4]{mhchem}
%%% Figuur programmeren
\usepackage{tikz}
\usetikzlibrary{positioning,calc,intersections,matrix}
\newcommand{\Receptor}[1]{%
\begin{scope}[rotate around={#1:(0,0)}]
    \draw[%
    ultra thick,
    smooth,
    line cap=round
    ] (-.50em,1em) -- ++(0,2.5em) 
    to[bend left] ++(-.5em,1.25em) to[bend left] ++ (-1.0em,1.25em);
%
    \draw[%
    xscale=-1,
    ultra thick,
    smooth,
    line cap=round
    ] (-.50em,1em) -- ++(0,2.5em) 
    to[bend left] ++(-.5em,1.25em) to[bend left] ++ (-1.0em,1.25em);
\end{scope}
}
\newcommand{\cellmembrane}[2]{%
        \node[fill=#1!45,
        minimum size=5em,
        inner sep=0,
        draw,
        circle,ultra thick
        ] (#2) at (0,0) {};
}
\newcommand{\nucleus}[2]{%
        \node[%
        shading=ball,
        ball color=#1!50,
        minimum size=2.5em,
        inner sep=0,
        draw,
        circle,
        ultra thick
        ] (#2) at (-0.20,0.20) {};
}
\newcommand{\flowblock}[4][]{%
    \node[%
        inner sep=5,
        minimum width=4cm,
        draw,
        fill=blue!25,
        align=center,
        rounded corners,
        #1
                     ]
                     (#2) {#3\\ ($n = #4$)};
}
\tikzset{>=latex}
\usepackage{graphicx}
%%% Marges bepaling
\usepackage{biblatex}
\addbibresource{Bronnen/bib.bib}
%%% Bibliografie
\usepackage[pdfusetitle]{hyperref}
\hypersetup{%
    colorlinks=true,
    citecolor=blue,
    pdfauthor={Edon Namani; Martijn Schuiling; Gijs Stuart; Casper Jansen; Martein Leen},
    pdftitle={Effect van borstvoeding op ontwikkelingskans van astma},
    pdfsubject={Immunologie; Allergie},
    pdfkeywords={Immunologie; Allergie; Borstvoeding; Astma; Odds ratio}
}
\usepackage[dutch]{cleveref}
%%% Hyperlinks & easy referencing
\usepackage[acronym,smallcaps]{glossaries}
\usepackage{scrlayer-scrpage}
\ihead{Borstvoeding \& astma}
\ifoot{\copyright{} 2019 \LaTeXe{} foundations}
\usepackage{threeparttable}
\addtokomafont{captionlabel}{\bfseries}
\KOMAoptions{DIV=calc}
%%%% acronyms
\newacronym{lovk}{lovk}{lange onverzadigde vetzuurketen}
\makeglossaries

\title{Effect van borstvoeding op astma incidentie}
\subtitle{Wetenschapsstage}
\author{Edon Namani\thanks{Praeses} \and Martijn Schuiling\thanks{Co\"ordinator} \and Gijs Stuart \and Casper Jansen \and Martein Leen}
\date{\today}
\subject{Immunologie \& allergie}
\begin{document}
\maketitle
    \begin{abstract}

    \end{abstract}

\section{Introductie}
In de afgelopen decennia is de incidentie van astma en andere type I hypersensitiviteit morbi ferm toegenomen \cite{Platts_Mills_2015}. Deze toegenomen incidentie veroorzaakt een verhoogde zorglast voor de maatschappij. Astma is een kinderlijke chronische respiratoire morbus. Meerdere factoren dragen bij aan de ontwikkeling van astma. De factoren zijn de genetische predispositie, de wijze van bevalling en het dieet \cite{abbas2017cellular,Houghteling_2015}.

Parallel met de stijging van de incidentie is het dieet van de zuigeling verandert. Borstvoeding was in het verleden gebruikelijker \cite{world1981contemporary,Victora_2016}. Het dieet is verschoven naar flesvoeding of vervroegd vaste voeding. In tegenstelling tot de twee laatstgenoemde diëten verschaft borstvoeding een modulerend werking op het immuunsysteem van de zuigeling.

\section{Methode}
    \subsection{Zoek \& selectie}

    \subsection{Standaardisering}

    %\subsection{Kwaliteitsbeoordeling}
\section{resultaten}

\section{Discussie \& conclusie}

\printbibliography 
\end{document}
