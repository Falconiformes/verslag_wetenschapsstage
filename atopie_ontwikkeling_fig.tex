\begin{figure}
    \centering
    \begin{tikzpicture}[ultra thick,auto,blok/.style={draw,thick,fill=blue!25}]
        \foreach \x in {0,120,240} {
        \Receptor{\x}
    }
        \cellmembrane{blue}{a}
        \nucleus{blue}{b}
        \node[left] at (a.north west) {T\textsubscript{h}1 cel};
    \begin{scope}[xshift=18em]
    \foreach \x in {0,120,240} {
    \Receptor{\x}
}
    \cellmembrane{red}{z}
    \nucleus{red}{b}
    \node[right] at (z.north east) {T\textsubscript{h}2 cel};
\end{scope}

    \begin{scope}[xshift=9em,yshift=9em]
    \foreach \x in {0,120,240} {
    \Receptor{\x}
}
    \cellmembrane{green}{r}
    \nucleus{green}{l}
    \node[right] at (r.north east) {Treg cel};
    \end{scope}
    \node[yshift=10em,blok] (tgfb) at (r) {TGF-$\beta$};
    \node[above of=tgfb,blok] (microbioom) {Divers microbioom};
    \node[above of=microbioom,blok] (lkvoz) {\glspl{lovk}};
    \node[blok,fill=red!25] (atopie) at (9em,-10em) {Atopie};
                    \begin{scope}[thick]
                        \draw[<->] (a) -- (z) node[midway] {differentiatie};
                        \draw[-|] (r.east) to[out=0,in=90] node{$\downarrow$ differentiatie} ($(z)+(0,2)$);
                        \draw[->] (r.west) to[out=180,in=90] node[anchor=south east]{$\uparrow$ differentiatie} ($(a)+(0,2)$);
                        \draw[-|] (a) to[bend right] node[anchor=north east] {vermindert}(atopie);
                        \draw[->] (z) to[bend left] node[auto]{verhoogt} (atopie);
                        \path[->] (lkvoz) edge (microbioom)
                                  (microbioom) edge (tgfb)
                                  (tgfb) edge ($(r)+(0,2)$)
                                  ;
                \end{scope}
    \end{tikzpicture}
    \captionbelow{Hypothetische werkingsmechanisme van borstvoeding op vermindering van atopie. \Glspl{lovk} verhoogt de diversiteit van het microbioom. Deze verhoging leidt tot een productieverhoging van de cytokine TGF-$\beta$. TGF-$\beta$ bevordert de groei van Tregs. Tregs hebben een immuunonderdrukkende functie en zorgen voor een hogere $\frac{\text{T\textsubscript{h}1}}{\text{T\textsubscript{h}2}}$. Tezamen veroorzaken zij een verlaagde predispositie op ontwikkeling van astma.}
    \label{fig:atopie_ontwikkeling}
\end{figure}
